\documentclass{article}
\usepackage{graphicx}
\usepackage{xepersian}
\usepackage{geometry}
\settextfont[Scale=1.2]{XB Zar}

% section numbering
\setcounter{secnumdepth}{3}
\renewcommand{\thesection}{\arabic{section}}
\renewcommand{\thesubsection}{\thesection.\arabic{subsection}}
\renewcommand{\thesubsubsection}{\thesection.\arabic{subsection}.\arabic{subsubsection}}

\title{ 
\begin{normalsize} به نام خدا \end{normalsize}
\\[7cm]
چکیده مقاله بیز ساده و رگرسیون لجستیک
\\[3cm]
}
\author{علیرضا نوریان
\\
\\ \small دانشگاه علم و صنعت ایران
\\ \small noorian@comp.iust.ac.ir
}

\begin{document}
\maketitle

\section{الگوریتم بیز ساده}
برای تخمین مقدار یک تابع باید آن را به صورت احتمال بروز مقادیر مختلف آن به شرط وقوع مقادیر معین در بردار ورودی نمایش دهیم. بیز ساده برای کاهش پیچیدگی محاسبه این احتمال فرض می‌کند که میان مقادیر بردار ورودی استقلال شرطی وجود دارد. یعنی به ازای خروجی مشخص، مقدار ورودی‌ها به هم وابسته نیست. این فرض اگرچه در موارد زیادی صحیح نیست، اما پیچیدگی تخمین را از حالت نمایی به حالت خطی تبدیل می‌کند.
احتمال بروز یک مقدار در خروجی با توجه به بردار وروی مشخص، طبق رابطه بیز، رابطه‌ی مستقیم با احتمال پیشینی بروز آن و احتمال بروز هر ویژگی با فرض وقوع خروجی و همچنین رابطه‌ی عکس با احتمال وقوع بردار ورودی دارد (رابطه ۲). ما به دنبال محتمل‌ترین مقدار خروجی برای بردار ورودی می‌گردیم و احتمال وقوع بردار ورودی برای همه‌ی خروجی‌ها یکسان و قابل صرف نظر است.
محاسبه‌ی احتمال پیشینی مقادیر خروجی به آسانی با محاسبه نسبت وقوع آن خروجی در داده‌های یادگیری بدست می‌آید. اما احتمال بروز هر ویژگی با فرض مشخص بودن خروجی نیازمند تحلیل بیشتری است. اگر بردار ورودی از مقادیر گسسته تشکل شده باشد، این احتمال برابر نسبت دفعات وقوع همزمان هر مقدار ورودی و خروجی مشخص به تعداد دفعات وقوع آن خروجی است (رابطه ۶).
پیوسته بودن بردار خروجی حل مساله را نیازمند فرضهای جدیدی می‌کند. توزیع گوسی همیشه انتخاب ما برای توزیع احتمال نامعلوم است که همان احتمال بروز ویژگی با فرض وقوع خروجی مشخص است. پارامترهای این توزیع احتمال برای هر ویژگی مستقلا محاسبه می‌شود. پارامتر میانگین این توزیع با میانگین مقدار ویژگی در نمونه‌هایی که خروجی مشخص‌شده را دارند و مقدار واریانس با محاسبه واریانس ویژگی در همان نمونه‌ها تقریب زده می‌شود (رابطه ۱۰  و ۱۱).

\section{رگرسیون لجستیک}
در این روش فرض می‌کنیم که احتمال مقادیر مختلف برای تابعی که قصد یادگیری آن را از روی ورودی‌ها داریم، از رابطه‌ی مشخصی (رابطه ۲۷) محاسبه می‌شود. فرض ما وابسته به برقراری شرایط بیز ساده نیست، اما با فرض برقرار بودن این شرایط و وجود توزیع گوسی برای بردار ورودی نیز این رابطه اثبات می‌شود و مقادیر پارامترهای موجود در آن به صورت تابعی از پارامترهای توزیع احتمال ورودی‌ها محاسبه می‌شود. راه عام محاسبه‌ی این پارامترها بدون در نظر گرفتن شرایط بیز ساده، بیشینه کردن احتمال وقوع داده‌های آموزشی در تابع احتمال مشخص شده است.
در واقع می‌توانیم فرمول محاسبه‌ی احتمال مقادیر مختلف خروجی را به هر حالتی فرض کنیم و بعد پارامترهای آن را طور تنظیم کنیم که مناسب داده‌های آموزشی شوند. مثلا می‌توانیم با استفاده از روش نزول روی بردار گرادیان تابع خطای پیش‌بینی خروجی از روی ورودی‌ها را کمینه کنیم. البته باید توجه داشته باشیم که تابع خطای مورد نظر باید به صورت محدب باشد تا رسیدن به جواب از این روش ممکن باشد. این منظور با استفاده از تابع رگرسیون لجستیک محقق می‌شود و از این بابت نگران استفاده از این تابع نیستیم.

\section{مقایسه میان دو روش}
به طور خلاصه می‌توان گفت در الگوریتم بیز ساده ما احتمال پیشینی وقوع خروجی و احتمال بروز بردار ورودی با فرض وقوع خروجی را محاسبه می‌کنیم تا به احتمال وقوع خروجی با فرض ورودی‌های مشخص برسیم در حالی که در رگرسیون لجستیک سعی می‌کنیم این مقدار را مستقیما تخمین بزنیم. وقتی شرایط بیز ساده و فرض توزیع گوسی متغیرهای بردار ورودی صادق باشد پاسخ هر دو روش با افزایش مجموعه‌ی آموزشی به یک مقدار میل می‌کند، اما هنگامی که این شرایط برقرار نیست پاسخ رگرسیون لجستک به انتظار ما نزدیک‌تر است؛ اگرچه سرعت رسیدن به پاسخ در روش بیز ساده بیشتر است. 

\end{document}
