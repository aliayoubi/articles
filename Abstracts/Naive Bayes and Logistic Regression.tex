\documentclass{article}
\usepackage{graphicx}
\usepackage{xepersian}
\usepackage{geometry}
\settextfont[Scale=1.2]{XB Zar}

% section numbering
\setcounter{secnumdepth}{3}
\renewcommand{\thesection}{\arabic{section}}
\renewcommand{\thesubsection}{\thesection.\arabic{subsection}}
\renewcommand{\thesubsubsection}{\thesection.\arabic{subsection}.\arabic{subsubsection}}

\title{ 
\begin{normalsize} به نام خدا \end{normalsize}
\\[7cm]
چکیده مقاله بیز ساده و تخمین منطقی
\\[3cm]
}
\author{علیرضا نوریان
\\
\\ \small دانشگاه علم و صنعت ایران
\\ \small noorian@comp.iust.ac.ir
}

\begin{document}
\maketitle

\section{الگوریتم بیز ساده}
برای تخمین مقدار یک تابع باید آن را به صورت احتمال بروز مقادیر مختلف آن به شرط وقوع مقادیر معین در بردار ورودی نمایش دهیم. بیز ساده برای کاهش پیچیدگی محاسبه این احتمال فرض می‌کند که میان مقادیر بردار ورودی استقلال شرطی وجود دارد. یعنی به ازای خروجی مشخص، مقدار ورودی‌ها به هم وابسته نیست. این فرض اگرچه در موارد زیادی صحیح نیست، اما پیچیدگی تخمین را از حالت نمایی به حالت خطی تبدیل می‌کند.
احتمال بروز یک مقدار در خروجی با توجه به بردار وروی مشخص، طبق رابطه بیز، رابطه‌ی مستقیم با احتمال پیشینی بروز آن و احتمال بروز هر ویژگی با فرض وقوع خروجی و همچنین رابطه‌ی عکس با احتمال وقوع بردار ورودی دارد (رابطه ۲). ما به دنبال محتمل‌ترین مقدار خروجی برای بردار ورودی می‌گردیم و احتمال وقوع بردار ورودی برای همه‌ی خروجی‌ها یکسان و قابل صرف نظر است.
محاسبه‌ی احتمال پیشینی مقادیر خروجی به آسانی با محاسبه نسبت وقوع آن خروجی در داده‌های یادگیری بدست می‌آید. اما احتمال بروز هر ویژگی با فرض مشخص بودن خروجی نیازمند تحلیل بیشتری است. اگر بردار ورودی از مقادیر گسسته تشکل شده باشد، این احتمال برابر نسبت دفعات وقوع همزمان هر مقدار ورودی و خروجی مشخص به تعداد دفعات وقوع آن خروجی است (رابطه ۶).
پیوسته بودن بردار خروجی حل مساله را نیازمند فرضهای جدیدی می‌کند. توزیع گوسی همیشه انتخاب ما برای توزیع احتمال نامعلوم است که همان احتمال بروز ویژگی با فرض وقوع خروجی مشخص است. پارامترهای این توزیع احتمال برای هر ویژگی مستقلا محاسبه می‌شود. پارامتر میانگین این توزیع با میانگین مقدار ویژگی در نمونه‌هایی که خروجی مشخص‌شده را دارند و مقدار واریانس با محاسبه واریانس ویژگی در همان نمونه‌ها تقریب زده می‌شود (رابطه ۱۰  و ۱۱).

\section{تخمین منطقی}


\end{document}
