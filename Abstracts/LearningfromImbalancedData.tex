\documentclass{article}
\usepackage{graphicx}
\usepackage{xepersian}
\usepackage{geometry}
\settextfont[Scale=1.2]{XB Zar}

% section numbering
\setcounter{secnumdepth}{3}
\renewcommand{\thesection}{\arabic{section}}
\renewcommand{\thesubsection}{\thesection.\arabic{subsection}}
\renewcommand{\thesubsubsection}{\thesection.\arabic{subsection}.\arabic{subsubsection}}

\title{ 
\begin{normalsize} به نام خدا \end{normalsize}
\\[7cm]
چکیده مقاله یادگیری از روی دادگان نامتوازن
\\[3cm]
}
\author{علیرضا نوریان
\\
\\ \small دانشگاه علم و صنعت ایران
\\ \small noorian@comp.iust.ac.ir
}

\begin{document}
\maketitle

\tableofcontents

\begin{abstract}

\end{abstract}

\section{مقدمه}
رشد حجم داده‌های موجود در زمینه‌های متخلف، موجب تمرکز فعالیتهای پژوهشی بر روی آنها شده است. بیشتر الگوریتم‌های تهیه شده برای کاوش متن، انتظار ورودی گرفتن داده‌هایی با توزیع استاندارد دارند. به همین دلیل در برخورد با داده‌های نامتوازن، به درستی عمل نمی‌کنند. این مساله، وقتی بیشتر مورد توجه قرار می‌گیرد که متوجه شیوع گسترده توزیع نامتوازن در داده‌های دنیای واقع هستیم.
\section{شرح مساله}
منظور از دادگان نامتوازن، توزیع کلاسها به صورت نامساوی و با اختلاف بسیار  زیاد مثل صد یا هزار براربر است. که بی‌تردید موجب به وجود آمدن تمایل بیش از حد در قضاوت می‌شود. بعضی از انواع دادگان مثل دادگان مربوط به یک بیماری نادر و یا نفوذ به یک سایت، به صورت ذاتی، نامتوازن هستند. در مقابل عواملی خارجی، مثل نقص در اندازه‌گیری‌ها و یا محدودیت‌های حجمی و یا زمانی مربوط به ذخیره‌سازی داده‌ها، موجب از بین رفتن توازن در دادگان می‌شوند.
عدم توازن در دادگان می‌تواند به صورت نسبی وجود داشته باشد. برای نمونه اگر از میان ۱۰۰۰۰ نمونه، فراوانی یکی از کلاسها ۱۰۰ برابر دیگری باشد، باید تقریبا ۱۰۰۰ نمونه از کلاس دیگر وجود دارد. در این شرایط اگرچه یادگیری کمی دچار خطا می‌شود، اما بیشتر عوامل مربوط به پیچیدگی مساله موثر هستند. در مقابل وقتی تعداد داده‌ها کم و یا تعداد ویژگی‌ها بسیار زیاد است، یادگیری شدیدا تحت تاثیر این مشکل قرار می‌گیرد.
\section{بهترین روشهای یادگیری نامتوازن}

\section{روشهای نمونه‌برداری}
دلیل اصلی استفاده از نمونه‌برداری در دادگان نامتوازن، برقراری توازن میان آنها است. البته بدون نمونه‌برداری هم یادگیری ممکن است ولی با این روش موجب بالا رفتن صحت در آزمایش‌ها می‌شود.

\subsection{نمونه‌برداری اتفاقی}
ساده‌ترین روش برای برقراری توازن، تکرار نمونه‌های نادر و یا حذف نمونه‌های فراوان به صورت اتفاقی و به اندازه‌ی لازم است. حذف نمونه‌های فراوان موجب حذف بعضی از شواهد تشخیص می‌شود و در مقابل تکرار نمونه‌های نادر، معیارهای تشخص را مختص همان نمونه‌ها می‌کند.
\subsubsection{حذف نمونه‌ها}
برای پرهیز از مشکلات ذکر شده، باید نمونه‌برداری را هوشمندانه‌تر انجام داد. الگوریتم EasyEnsemble با همین هدف، به طور اتفاقی مجموعه‌هایی را از داده‌های اکثریت انتخاب کرده و برای تفکیک هر کدام از آنها و مجموعه داده‌های اقلیت، یک ابزار رده‌بندی\LTRfootnote{Classifier} تولید می‌کند. BalanceCascade عنوان الگوریتم دیگری برای انجام همین کار است. این الگوریتم در هر نوبت اجرا، مجموعه‌ای به تعداد داده‌های اقلیت از داده‌های اکثریت جدا کرده و ابزاری برای رده‌بندی آنها ایجاد می‌کند. سپس با استفاده از همین ابزار مجموعه‌ی حاصل از ترکیب داده‌های جدا شده و داده‌های اقلیت را رده‌بندی می‌کند. داده‌هایی که در این بخش در رده‌ی اکثریت قرار گیرند، توسط ابزار رده‌بندی شناخته شده‌اند، پس آنها را از مجموعه‌ی داده‌های اصلی حذف می‌کند. ادامه دادن این رویه موجب تولید یک ابزار مناسب برای رده‌بندی می‌شود.
همچنین روشهایی برای کاهش آگاهانه‌ی داده‌های اکثریت با استفاده از الگوریتم KNN پیشنهاد شده است. برای نمونه در یکی از این روشها، نمونه‌هایی از داده‌های اکثریت را حذف می‌کنیم که فاصله‌ی آنها از دورترین نمونه‌های اقلیت، کوتاهتر باشد و به این ترتیب مرز میان دو گروه واضح‌تر می‌شود.

\subsubsection{تولید نمونه}
افزودن نمونه به مجموعه‌ی اقلیت نیز با روشهای مختلفی انجام می‌شود. الگورتیم SMOTE که از انواع موفق در این روش است، سعی می‌کند نمونه‌هایی را شبیه به نمونه‌های موجود تولید کند. برای این کار، دو نمونه نزدیک به هم را به طور اتفاقی انتخاب کرده و نمونه‌ای بین آنها (با ترکیب ویژگی‌های آنها) ایجاد می‌کند.
شکل تکامل یافته‌تری از این فرایند، به محدوده‌های نیازمند تولد نمونه توجه می‌کند. در این روش نمونه‌ی جدید در میان نمونه‌هایی از اقلیت ایجاد می‌شوند که در مجاورت تعداد قابل توجهی نمونه‌ی اکثریت قرار گرفته‌اند. به این ترتیب نمونه‌هایی که بیشتر احتمال خطا در مورد آنها وجود دارد، تقویت می‌شوند. بعد از تولید نمونه با حذف نمونه‌های مرزی داده‌های اکثریت و اقلیت، مجموعه‌ی داده‌ها بسیار تمیزتر شده و ابزار رده‌بندی در این حالت راحت‌تر قضاوت می‌کند.
روش دیگر برای تولید نمونه، استفاده از خوشه‌بندی است. در این روش ابتدا داده‌ها خوشه‌بندی می‌شوند و مرکز هر خوشه مشخص می‌شود. بعد از این مرحله باید خوشه‌های اقلیت به اندازه‌ی خوشه‌های اکثریت پر شوند. با توجه به مشخص بودن مرز میان خوشه‌ها، تولید نمونه اتفاقی در محدوده‌ی هر خوشه بسیار آسان است.
تولید نمونه از حذف نمونه پیچیده‌تر است و نیاز به محاسبه‌ی بیشتری هم دارد. به همین دلیل روشهایی ترکیبی ایجاد شده‌اند که هم از ایده‌ی انتخاب‌های متعدد از مجموعه‌ی اکثریت استفاده می‌کنند و هم با تکرار داده‌های اقلیت، سعی می‌کنند تشخیص آنها را آسانتر کنند.

\section{معیارهای ارزیابی برای روشهای یادگیری نامتوازن}

\section{فرصت‌ها و چالش‌ها}

\end{document}















