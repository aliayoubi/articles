\documentclass{article}
\usepackage{graphicx}
\usepackage{xepersian}
\usepackage{geometry}
\settextfont[Scale=1.2]{XB Zar}

% section numbering
\setcounter{secnumdepth}{3}
\renewcommand{\thesection}{\arabic{section}}
\renewcommand{\thesubsection}{\thesection.\arabic{subsection}}
\renewcommand{\thesubsubsection}{\thesection.\arabic{subsection}.\arabic{subsubsection}}

\title{ 
\begin{normalsize} به نام خدا \end{normalsize}
\\[7cm]
چکیده مقاله یادگیری از روی دادگان نامتوازن
\\[3cm]
}
\author{علیرضا نوریان
\\
\\ \small دانشگاه علم و صنعت ایران
\\ \small noorian@comp.iust.ac.ir
}

\begin{document}
\maketitle

\tableofcontents

\begin{abstract}

\end{abstract}

\section{مقدمه}
رشد حجم داده‌های موجود در زمینه‌های متخلف، موجب تمرکز فعالیتهای پژوهشی بر روی آنها شده است. بیشتر الگوریتم‌های تهیه شده برای کاوش متن، انتظار ورودی گرفتن داده‌هایی با توزیع استاندارد دارند. به همین دلیل در برخورد با داده‌های نامتوازن، به درستی عمل نمی‌کنند. این مساله، وقتی بیشتر مورد توجه قرار می‌گیرد که متوجه شیوع گسترده توزیع نامتوازن در داده‌های دنیای واقع هستیم.
\section{شرح مساله}
منظور از دادگان نامتوازن، توزیع کلاسها به صورت نامساوی و با اختلاف بسیار  زیاد مثل صد یا هزار براربر است. که بی‌تردید موجب به وجود آمدن تمایل بیش از حد در قضاوت می‌شود. بعضی از انواع دادگان مثل دادگان مربوط به یک بیماری نادر و یا نفوذ به یک سایت، به صورت ذاتی، نامتوازن هستند. در مقابل عواملی خارجی، مثل نقص در اندازه‌گیری‌ها و یا محدودیت‌های حجمی و یا زمانی مربوط به ذخیره‌سازی داده‌ها، موجب از بین رفتن توازن در دادگان می‌شوند.
عدم توازن در دادگان می‌تواند به صورت نسبی وجود داشته باشد. برای نمونه اگر از میان ۱۰۰۰۰ نمونه، فراوانی یکی از کلاسها ۱۰۰ برابر دیگری باشد، باید تقریبا ۱۰۰۰ نمونه از کلاس دیگر وجود دارد. در این شرایط اگرچه یادگیری کمی دچار خطا می‌شود، اما بیشتر عوامل مربوط به پیچیدگی مساله موثر هستند. در مقابل وقتی تعداد داده‌ها کم و یا تعداد ویژگی‌ها بسیار زیاد است، یادگیری شدیدا تحت تاثیر این مشکل قرار می‌گیرد.
\section{بهترین روشهای یادگیری نامتوازن}

\section{معیارهای ارزیابی برای روشهای یادگیری نامتوازن}

\section{فرصت‌ها و چالش‌ها}

\end{document}
